\def\firstname{firstname}
\def\lastname{lastname}
\def\aufgabenblatt{1}
\include{../common/header.tex}

\begin{document}

\thispagestyle{page1} 

\section{Vorbereitung}

\subsection{Software installations}

Install the required software on a workstation of your choice. You need for the lecture:

\begin{enumerate}
\item A git client, e.g. turtoise git under Windows or git under Linux.
\item Python 3.11 or higher 
\item A Latex installation, e.g. miktex and texworks under Windows or a corresponding distribution under Linux
\item An installation of Jupyter Lab
\end{enumerate}

\subsection{Create and Register Repository (5 points)}

Check out the exercise file \texttt{Exercise\_01.zip} from the canvas course. During the semester, the assignment sheets as well as the associated data and sample solutions will be checked into the course. 

I needed the following time to complete the task:

\section{Working with Images}

\subsection{File Input and Output for Images (10 Points)}

Now edit the Jupyter Notebook \texttt{File\_Access.ipynb}. The notebook loads a grayscale image and displays it.

\begin{enumerate}
\item[a)] Extend the image, such that it stores the file withoutout midifications under the name \texttt{result\_a.png} in the direcory of the notebook. 
\item[b)] Explain the dimensions of the loaded data: what are rows and columns, and how do you index a specific pixel?
\item[c)] Change the notebook to load the color version of the image. Display the color version.
\item[b)] Explain the dimensions of the loaded data: what are rows and columns, and how do you index a specific pixel? How are color channels treated?
\end{enumerate}

I needed the following time to complete the task:

\subsection{Pixel Access (10 Points)}

Now edit the Jupyter Notebook \texttt{Pixel\_Access.ipynb}. 

\begin{enumerate}
\item[a)] Modify the image by creating a black border. The border should frame the image on all four sides, and be three pixels wide. Store the image under the name \texttt{result\_b.png}
\item[b)] Now create a border that is 10 pixels wide, but semi-transparant. You can achieve transparency by blending pixel values. Store the result with the file name \texttt{result\_c.png}
\end{enumerate}

I needed the following time to complete the task:

\end{document}